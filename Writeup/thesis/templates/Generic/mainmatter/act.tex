\chapter[On Autcorrelation Tables]{On Autocorrelation Tables and Differential-Linear Cryptanalysis}\label{ch:act}
\openepigraph{Attacks only get better, never get worse.}{Unknown}

\vspace*{-\baselineskip}
\newthought{Synopsis}\hspace{1.5em}
We observe that the \DLCT/ coincides with the \ACT/ and provide a detailed analysis of it.
This chapter is based on the preprint
\begin{quote}
    \fullfullcite{EPRINT:CanKolWie19}.
\end{quote}
All authors contributed equally.

After discussing the equivalence of \DLCT/ and \ACT/, we analyse a link between the \ACT/, \DDT/ and \LAT/, which corresponds to the well known relation between differential and linear cryptanalysis already exhibited by \textcite{EC:ChaVau94} and later by \textcite{EC:BloNyb13}.
We then show that the (extended) \ACT/ spectrum is invariant under (extended) affine equivalence, but not under CCZ equivalence.
The last observation is based on the fact that the \ACT/ of a function and that of its inverse can behave quite differently.
This is of particular interest, as it might lead to the case where a differential-linear attack work better for the decryption than the encryption of a block cipher.
It is also in stark contrast to standard differential or linear cryptanalysis, which both behave the same in both directions.

We extend our analysis of the \ACT/ by exhibiting a lower bound on the absolute indicator for generic vectorial Boolean functions.
Eventually we prove some more specific properties for some specific types of vectorial functions.

\section{On the DLCT and ACT}%
\label{sec:act:dlct}

Let us define the \DLCT/ as follows.
\begin{definition}[\DLCTl/]
    Given a permutation $F : \F_2^n \to \F_2^n$, the corresponding \emph{\DLCTf/} consists of the following elements:
    \begin{equation}
        \DLCT/_F[\alpha, \beta] \coloneqq
        \sum_{x \in \F_2^n} \parens{-1}^{\iprod{\beta}{(F(x) + F(x + \alpha))}}
    \end{equation}
    We leave out the subscript, if it is clear from the context.
\end{definition}
While \textcite{EPRINT:dlct19} defined the entry at position $(\alpha, \beta)$ as
\begin{equation*}
    \DLCT/_F[\alpha, \beta] = \abs{\set{x \given \iprod{\beta}{(F(x) + F(x + \alpha))} = 0}} - 2^{n-1},
\end{equation*}
it is easy to see that both definitions only differ in a factor of $2$ for each entry:
\begin{align*}
    &\ 2 \cdot \parens{\abs{\set{x \in \F_2^n \given \iprod{\beta}{(F(x) + F(x + \alpha))} = 0}} - 2^{n-1}}
     = \abs{M_0} + \abs{M_0} - 2^n \\
    &= \abs{M_0} + \parens{2^n - \abs{M_1}} - 2^n = \abs{M_0} - \abs{M_1}
     = \sum_{x \in \F_2^n} \parens{-1}^{\iprod{\beta}{(F(x) + F(x + \alpha))}},
\end{align*}
where we define $M_i = \set{x \in \F_2^n \given \iprod{\beta}{(F(x) + F(x+\alpha))} = i}$ for the sake of readability.

Our first observation on the \DLCT/ is that it basically contains the autocorrelation spectra of the component functions of~\(F\).
Recall that the \emph{autocorrelation of $f$} is defined as, see \eg/~\textcite[277]{BMM:Carlet10a},
\begin{equation*}
    \AC{f}(\alpha) \coloneqq \FC{\derive{\alpha}{f}}(0).
\end{equation*}
Similar to Walsh coefficients, this notion can naturally be generalised to vectorial Boolean functions as
\begin{equation*}
    \AC{F}(\alpha, \beta) \coloneqq \AC{F_\beta}(\alpha) = \FC{\derive{\alpha}{F_\beta}}(0),
\end{equation*}
and we name the $\AC{F}(\alpha, \beta)$ the \emph{autocorrelation coefficients} of $F$.
In other words, the autocorrelation coefficients of a vectorial Boolean function consist of the autocorrelation of its component functions.
To easily see the correspondence, we only have to conclude that
\begin{equation*}
    \FC{\derive{\alpha}{F_\beta}}(0) = \sum_{x}\parens{-1}^{\iprod{\beta}{(F(x) + F(x+\alpha))}}.
\end{equation*}
\textcite[Section~3]{DCC:ZhaZheIma00} introduced the term \ACTf/ for a vectorial Boolean function which, analogously to the Walsh coefficient and \LAT/ again, contains the autocorrelation spectra of $F$'s component functions.
This implies for the \DLCT/ that
\begin{equation*}
    \DLCT/_F[\alpha, \beta] = \AC{F}(\alpha, \beta) = \ACTs/_F[\alpha, \beta].
\end{equation*}
For the remainder of this paper we thus stick to the established notion of the autocorrelation.

\textcite{JUCS:ZhaZhe96} termed the \emph{absolute indicator} $\absoluteindicator{f}$ as the maximum absolute value of the autocorrelation (of a Boolean function $f$).
Analogously for a vectorial Boolean function, we define
\begin{definition}[Absolute indicator]
    Given a function $F : \F_2^n \to \F_2^n$.
    The \emph{absolute indicator} of $F$ is
    \begin{equation*}
        \absoluteindicator{F} \coloneqq \max_{\beta \in \F_2^n \setminus \set{0}} \absoluteindicator{F_\beta} = \max_{\alpha, \beta \in \F_2^n \setminus \set{0}} \abs{\AC{F}(\alpha, \beta)}\;,
    \end{equation*}
    that is the maximum absolute indicator of $F$'s non-trivial component functions.
\end{definition}

We call the multiset $\set{\AC{F}(\alpha, \beta) \given \alpha, \beta \in \F_2^n}$ the \emph{autocorrelation} or \ACT/ \emph{spectrum} of $F$.

\textcite[Section~3]{DCC:ZhaZheIma00} further showed that
\begin{equation*}
    \ACTs/ = \DDTs/ \cdot H
\end{equation*}
where $H$ is the Walsh matrix of order $n$.
In other words, the \ACT/ is the Walsh transformed \DDT/ of $F$:
\begin{align*}
    (\DDT/_F \cdot H)[\alpha, \beta]
    &= \sum_{\gamma \in \F_2^n} \abs{\set{x \in \F_2^n \given F(x) + F(x+\alpha) = \gamma}} \cdot H[\gamma, \beta] \\
    &= \sum_{\gamma \in \F_2^n} \abs{\set{x \in \F_2^n \given F(x) + F(x+\alpha) = \gamma}} \cdot \parens{-1}^{\iprod{\gamma}{\beta}} \\
    &= \sum_{x \in \F_2^n} \parens{-1}^{\iprod{\beta}{(F(x) + F(x+\alpha))}} \\
    &= \AC{F}(\alpha, \beta)\;.
\end{align*}
Because of the correspondence between the \ACT/ and the \DLCT/, this corresponds to~\cite[Proposition~1]{EC:BDKW19}.

Let us now recall some properties of and links between the above discussed notations.

\subsection{Links between the ACT, DDT, and Walsh transformation}

\newthoughtpar{Sum of the ACT entries, within a row or a column}
It is well-known that the entries $\AC{F}(\alpha,\beta)$, $\beta \neq 0$ in each nonzero row in the \ACT/ of~$F$ sum to zero if and only if $F$ is a permutation (see \eg/~\cite[Proposition~2]{TIT:BCCL06}).
The same property holds when the entries $\AC{F}(\alpha,\beta)$, $\alpha \neq 0$ in each nonzero column in the \ACT/ are considered (see \eg/~\cite[Eq.~(9)]{TIT:BCCL06}).

\newthoughtpar{Link between differential and linear cryptanalysis}
The following proposition shows that the restriction of the autocorrelation function $\alpha \mapsto \AC{F}(\alpha,\beta)$ can be seen as the discrete Fourier transform of the squared Walsh transform of $F_\beta$: $\gamma \mapsto \FC{F}^2(\gamma,\beta)$.
As previously mentioned, $\beta \mapsto \AC{F}(\alpha,\beta)$ similarly corresponds to the Fourier transform of the row of index~$\alpha$ in the \DDT/: $\theta \mapsto \DC{F}(\alpha,\theta)$.
It is worth noticing that this correspondence points out the well-known relationship between the Walsh transform of~$F$ and its \DDT/, which was exhibited in~\citeonly{EC:ChaVau94,EC:BloNyb13}.

\begin{proposition}
  Let $F$ be a function from $\F_2^n$ to $\F_2^n$.
  Then, for all $\alpha,\beta \in \F_2^n$, we have
  \begin{align}
      \AC{F}(\alpha,\beta) &= 2^{-n} \sum_{\gamma \in \F_2^n} (-1)^{\iprod{\alpha}{\gamma}} \FC{F}^2(\gamma,\beta) \label{eq1}\\
               &= \sum_{\theta \in \F_2^n} (-1)^{\iprod{\beta}{\gamma}} \DC{F}(\alpha,\theta) \label{eq2}\;.
  \end{align}
  Conversely, the inverse Fourier transform leads to
  \begin{align}
       \FC{F}^2(\alpha,\beta) &= \sum_{\gamma \in \F_2^n} (-1)^{\iprod{\alpha}{\gamma}} \AC{F}(\gamma,\beta) \label{eq3}\\
       \DC{F}(\alpha,\beta) &= 2^{-n} \sum_{\theta \in \F_2^n} (-1)^{\iprod{\beta}{\theta}} \AC{F}(\alpha,\theta) \label{eq4}\;,
  \end{align}
  for all $\alpha,\beta \in \F_2^n$.
\end{proposition}
\begin{proof}
    We first prove Relation~\eqref{eq1} which involves the squared Walsh transform.
    For all $\alpha,\beta \in \F_2^n$, we have
    \begin{align*}
        \sum_{\gamma \in \F_2^n} (-1)^{\iprod{\alpha}{\gamma}} \FC{F}^2(\gamma,\beta)
        &= \sum_{\gamma \in \F_2^n} (-1)^{\iprod{\alpha}{\gamma}}  \sum_{x \in \F_2^n} (-1)^{F_\beta(x) + \iprod{\gamma}{x}} \sum_{y \in \F_2^n} (-1)^{F_\beta(y) + \iprod{\gamma}{y}} \\
        &= \sum_{x,y \in \F_2^n} (-1)^{F_\beta(x)+F_\beta(y)} \parens{\sum_{\gamma \in \F_2^n} (-1)^{\iprod{\gamma}{(\alpha+x+y)}}}\\
        &= 2^n \sum_{x \in \F_2^n} (-1)^{F_\beta(x)+F_\beta(x+\alpha)} = 2^n \AC{F}(\alpha,\beta)
    \end{align*}
    where the last equality comes from the fact that
    \begin{equation*}
        \sum_{\gamma \in \F_2^n} (-1)^{\iprod{\gamma}{(\alpha+x+y)}} = \begin{cases*}
            2^n & for $\alpha+x+y = 0$\\
            0   & otherwise.
        \end{cases*}
    \end{equation*}
    Obviously, \cref{eq3} can be directly derived from \cref{eq1} by applying the inverse Fourier transform.
    We now prove the relation involving the \DDT/, namely \cref{eq4}.
    For all $\alpha,\beta \in \F_2^n$, we have
    \begin{align*}
        \sum_{\theta \in \F_2^n} (-1)^{\iprod{\beta}{\theta}} \AC{F}(\alpha,\theta)
        &= \sum_{\theta \in \F_2^n} (-1)^{\iprod{\beta}{\theta}} \sum_{x \in \F_2^n} (-1)^{\iprod{\theta}{\derive{\alpha}{F}(x)}}\\
        &= \sum_{x \in \F_2^n} \sum_{\theta \in \F_2^n} (-1)^{\iprod{\theta}{(\beta + \derive{\alpha}{F}(x)))}}\\
        &= \sum_{x \in \derive{\alpha}{F}^{-1}(\beta)} 2^n = 2^n \DC{F}(\alpha,\beta)\;.
    \end{align*}
    \cref{eq2} then follows directly by the inverse Fourier transform.
\end{proof}

As a corollary, Parseval's equality leads to an expression of the sum of all squared entries in each row, and in each column of the autocorrelation table.

\begin{corollary}\label{cor:Parseval}
    Let $F$ be a function from $\F_2^n$ to $\F_2^n$.
    Then, for all $\alpha,\beta \in \F_2^n$, we have
    \begin{equation*}
        \sum_{\alpha \in \F_2^n} \AC{F}^2(\alpha,\beta) = 2^{-n} \sum_{\gamma \in \F_2^n} \FC{F}^4(\gamma,\beta) \quad \text{and} \quad \sum_{\beta \in \F_2^n} \AC{F}^2(\alpha,\beta) = 2^n \sum_{\theta \in \F_2^n} \DC{F}^2(\alpha,\theta)\;.
    \end{equation*}
\end{corollary}

In the following, we show that the \ACT/ spectrum is affine invariant, the extended \ACT/ spectrum is extended affine invariant and the \ACT/ spectrum is not invariant under CCZ equivalence.

\subsection{Invariance under Equivalence Relations}%
\label{act:sec:invariance}

Having an equivalence relation on the set of all $n$-bit functions, allows us to partition these functions into equivalence classes.
Properties which are invariant under this equivalence notion can then be tested on only one representative of each class -- resulting in a massive decrease of complexity, if we want to characterise the whole set of functions under this property.
Three well-known equivalences are the following:
\begin{definition}[Equivalence relations for vectorial Boolean functions]
    Given two functions $F$ and $G : \F_2^n \to \F_2^n$.
    We say these functions are
    \begin{enumerate}
        \item \emph{affine equivalent} ($F \aequiv G$) if there exist two affine permutations $A$ and $B$ such that $G = B \circ F \circ A$,
        \item \emph{extended affine equivalent} ($F \eaequiv G$) if there exist two affine permutations $A$ and $B$, and an affine function $C$ such that $G = B \circ F \circ A + C$,
        \item \emph{CCZ equivalent} ($F \cczequiv G$) if their graphs are affine equivalent.
    \end{enumerate}
    Here, the graph of a function is defined as $\set{(x, F(x)) \given x \in \F_2^n}$.
\end{definition}

A nice property of the \ACT/ is that its spectrum is invariant under affine equivalence, and further its extended \ACT/ spectrum, that is the multiset of absolute autocorrelation coefficients, $\set*{\abs*{\AC{F}(\alpha, \beta)} \given \alpha, \beta \in \F_2^n}$, is invariant under extended affine equivalence.

\begin{proposition}[Affine and Extended Affine Invariance]\label{act:prop:affine}
    Given two permutations $F$, and $G$ on $\F_2^n$.
    Let further $A = L_a + a$, $B = L_b + b$ be two affine permutations and $C = L_c + c$ be an affine function.
    Then
    \begin{align*}
        G = B \circ F \circ A     \quad &\Rightarrow \quad      \AC{G}(\alpha, \beta)  =      \AC{F}(L_a(\alpha), L_b^\top(\beta))
        \intertext{and}
        G = B \circ F \circ A + C \quad &\Rightarrow \quad \abs{\AC{G}(\alpha, \beta)} = \abs{\AC{F}(L_a(\alpha), L_b^\top(\beta))}\;,
    \end{align*}
    where $L^\top$ denotes the adjoint of the linear mapping~$L$.
\end{proposition}
\begin{proof}
    From the affine equivalence of $F$ and $G$, we have
    \begin{equation*}
        G(x) = L_b(F(L_a(x) + a)) + b\;.
    \end{equation*}
    Each entry $\ACT/_F[\alpha, \beta]$ corresponds to the number of solutions $x$ for the equation
    \begin{equation*}
        \iprod{\beta}{(F(x) + F(x + \alpha))} = 0.
    \end{equation*}
    Thus for each entry of $G$'s \ACT/ at position $(\alpha, \beta)$, we count the number of solutions for
    \begin{align*}
        \iprod{\beta}{\bracket{L_b(F(L_a(x) + a)) + b + L_b(F(L_a(x + \alpha) + a)) + b}} &= 0.
        \intertext{Substituting $x^\prime = L_a(x) + a$, this simplifies to}
        \iprod{\beta}{L_b(F(x^\prime) + F(x^\prime + L_a(\alpha)))} &= 0 \\
        \Leftrightarrow \quad \iprod{L_b^\top(\beta)}{(F(x^\prime) + F(x^\prime + L_a(\alpha)))} &= 0
    \end{align*}
    thus the number of solutions for this equation is nothing else as the \ACT/ entry of $F$ at position $(L_a(\alpha), L_b^\top(\beta))$.

    For the second point, we now only have to show how $G = F + C$ behaves.
    By definition of the autocorrelation we have
    \begin{align*}
        \AC{G}(\alpha, \beta)
        &= \sum_{x \in \F_2^n} \parens{-1}^{\iprod{\beta}{(G(x) + G(x+\alpha))}} \\
        &= \sum_{x \in \F_2^n} \parens{-1}^{\iprod{\beta}{(F(x) + L_c(x) + c + F(x+\alpha) + L_c(x+\alpha) + c)}} \\
        \intertext{where $C(x) = L_c(x) + c$}
        &= \sum_{x \in \F_2^n} \parens{-1}^{\iprod{\beta}{(F(x) + F(x+\alpha) + L_c(\alpha))}} \\
        &= \parens{-1}^{\iprod{\beta}{(L_c(\alpha))}} \AC{F}(\alpha, \beta),
    \end{align*}
    and thus the affine map $C$ only influences the sign of the autocorrelation value.
\end{proof}

The (extended) affine invariance of the (extended) \ACT/ spectra follows directly from this proposition.
\begin{corollary}
    Given two permutations $F$, and $G$ on $\F_2^n$.
    \begin{itemize}
        \item If $F \aequiv G$, the \ACT/ spectrum of $F$ equals that of $G$.
        \item If $F \eaequiv G$, the extended \ACT/ spectrum of $F$ equals that of $G$.
    \end{itemize}
\end{corollary}

To examine the behaviour under CCZ equivalence, let us first recall that the \ACT/ is related to linear structures in the following way, see also~\citeonly{DCC:ZhaZheIma00,LIGHTSEC:MakTez14}.
\begin{lemma}[Linear Structures]\label{act:lem:ls}
    Given $F : \F_2^n \to \F_2^n$, then
    \begin{equation*}
        \AC{F}(\alpha, \beta) = \pm 2^{n}
    \end{equation*}
    if and only if $(\alpha, \beta)$ forms a linear structure for $F$.
\end{lemma}
\begin{proof}
    This follows from the fact that, for a linear structure, by definition,
    \begin{equation*}
        \iprod{\beta}{(F(x) + F(x+\alpha))}
    \end{equation*}
    is constant zero or one.
    The sign of the entry thus determines, if the linear structure is constant one (negative) or zero (positive).
\end{proof}

One consequence of this is the next corollary.
\begin{corollary}[Inversion]\label{cor:act:inversion}
    Given a permutation $F : \F_2^n \to \F_2^n$, then the \ACT/ spectrum of $F$ is in general not equal to the \ACT/ spectrum of $F^{-1}$.
\end{corollary}
\begin{proof}
    Counterexamples are the S-boxes used in the ciphers \safer/~\citeonly{FSE:Massey93}, \sc2000/~\citeonly{FSE:SYYTIYTT01}, and \fides/~\citeonly{CHES:BBKMW13}, where the S-boxes have linear structures in one direction but non in the other direction, and the Gold permutations as analysed in \cref{sec:APN}.
\end{proof}
An interesting implication of this is that it might be advantageous when doing a differential-linear cryptanalysis, to look at both directions, encryption and decryption, of a cipher.

Another direct consequence of \cref{cor:act:inversion} is the following result.
\begin{corollary}
    Given two CCZ-equivalent permutations $F \cczequiv G$ on $\F_2^n$.
    Their \ACT/ spectrum is in general not invariant.
\end{corollary}
\begin{proof}
    A function and its inverse are always CCZ equivalent.
    Thus \cref{cor:act:inversion} gives a counterexample.
\end{proof}

\citeauthor{DCC:ZhaZheIma00} further showed how the \ACT/ of $F$ and its inverse $F^{-1}$ are related, see~\cite[Corollary~1]{DCC:ZhaZheIma00}.
In particular they showed that
\begin{equation*}
    \ACT/_{F^{-1}} = H^{-1} \cdot \ACT/_{F}^\top \cdot H, % \label{act:eq:act_matr_inverse}
\end{equation*}
which in our notation is
\begin{equation*}
    \AC{F^{-1}}(\alpha, \beta) = \frac{1}{2^n} \sum_{\gamma, \theta \in \F_2^n} \parens{-1}^{\iprod{\alpha}{\gamma} + \iprod{\beta}{\theta}} \AC{F}(\gamma, \theta).
\end{equation*}

\section{Lower bound on the absolute indicator}%
\label{sec:act:absolute_indicator}

Finding the smallest possible absolute indicator for a \emph{Boolean function} is an open question investigated by many authors.
Zhang and Zheng conjectured~\cite[Conjecture 1]{JUCS:ZhaZhe96} that the absolute indicator of a balanced Boolean function of $n$~variables was at least $2^{\frac{n+1}{2}}$.
But this was later disproved first for odd values of~$n \geqslant 9$ by modifying the Patterson-Wiedemann construction, namely for $n \in \set{9, 11}$ in~\citeonly{TIT:KavMaiYuc07}, for $n=15$ in~\citeonly{TIT:MaiSar02,DAM:Kavut16} and for $n=21$ in~\citeonly{DM:GanKesMai06}.
For the case $n$ even, \citeauthor{TIT:TanMai18}~\citeonly{TIT:TanMai18} gave a construction for balanced Boolean functions with absolute indicator strictly less than $2^{n/2}$ when $n \equiv 2 \bmod{4}$.
Very recently, similar examples for $n \equiv 0 \bmod{4}$ were exhibited by~\citeauthor{DCC:KavMaiTan19}~\citeonly{DCC:KavMaiTan19}.
However, we now show that such small values for the absolute indicator cannot be achieved for \emph{vectorial Boolean functions}.

\subsection{General Case}

Parseval's equality leads to the following lower bound on the sum of all \emph{squared} autocorrelation coefficients in each row.
This result can be found in~\cite{FSE:Nyberg94} (see also \cite[Theorem~2]{TIT:BCCL06}), but we recall the proof for the sake of completeness.
\begin{proposition}\label{prop:boundAPN}
    Let $F$ be a function from $\F_2^n$ into $\F_2^n$.
    Then, for all $\alpha \in \F_2^n$, we have
    \begin{equation*}
        \sum_{\beta \in \F_2^n \setminus \set{0}} \AC{F}^2(\alpha, \beta) \geqslant 2^{2n}\;.
    \end{equation*}
    Moreover, equality holds for all nonzero~$\alpha \in \F_2^n$ if and only if $F$ is \APN/.
\end{proposition}
\begin{proof}
    From~\cref{cor:Parseval}, we have that, for all $\alpha \in \F_2^n$,
    \begin{equation*}
        \sum_{\beta \in \F_2^n} \AC{F}^2(\alpha, \beta) = 2^n \sum_{\theta \in \F_2^n} \DC{F}^2(\alpha, \theta)
    \end{equation*}
    Cauchy-Schwarz inequality implies that
    \begin{equation*}
        \parens{\sum_{\theta \in \F_2^n} \DC{F}(\alpha, \theta)}^2 \leqslant \parens{\sum_{\theta \in \F_2^n} \DC{F}^2(\alpha, \theta)} \cdot \abs{\set{\theta \in \F_2^n \given \DC{F}(\alpha, \theta) \neq 0}}\;,
    \end{equation*}
    with equality if and only if all nonzero elements in $\set{\DC{F}(\alpha, \theta) \given \theta \in \F_2^n}$ are equal.
    Using that
    \begin{equation*}
        \abs{\set{\theta \in \F_2^n \given \DC{F}(\alpha, \theta) \neq 0}} \leqslant 2^{n-1}
    \end{equation*}
    with equality for all nonzero $\alpha$ if and only if $F$ is \APN/, we deduce that
    \begin{equation*}
        \sum_{\theta \in \F_2^n} \DC{F}^2(\alpha, \theta) \geqslant 2^{2n} \times 2^{-(n-1)} = 2^{n+1}
    \end{equation*}
    again with equality for all nonzero $\alpha$ if and only if $F$ is \APN/.
    Equivalently, we deduce that
    \begin{equation*}
        \sum_{\beta \in \F_2^n} \AC{F}^2(\alpha, \beta) \geqslant 2^{2n+1}
    \end{equation*}
    with equality for all nonzero $a$ if and only if $F$ is \APN/.
    Then the result follows from the fact that
    \begin{equation*}
        \sum_{\beta \in \F_2^n} \AC{F}^2(\alpha, \beta) = 2^{2n} + \sum_{\beta \in \F_2^n \setminus \set{0}} \AC{F}^2(\alpha, \beta)\;.
    \end{equation*}
\end{proof}

From the lower bound on the sum of all squared coefficients within a row of the \ACT/, we deduce the following lower bound on the absolute indicator.
\begin{proposition}[Lowest possible absolute indicator] \label{prop:lowerbound}
    Let $F$ be a function from $\F_2^n$ into $\F_2^n$.
    Then,
    \begin{equation*}
        \absoluteindicator{F} \geqslant \frac{2^n}{\sqrt{2^n-1}} > 2^{n/2}\;.
    \end{equation*}
\end{proposition}
\begin{proof}
    From the facts that
    \begin{equation*}
        \sum_{\beta \in \F_2^n \setminus \set{0}} \AC{F}^2(\alpha, \beta) \geqslant 2^{2n}
    \end{equation*}
    and
    \begin{equation*}
        \sum_{\beta \in \F_2^n \setminus \set{0}} \AC{F}^2(\alpha, \beta) \leqslant \absoluteindicator{F}^2 (2^n-1)
    \end{equation*}
    the result directly follows.
\end{proof}

We can get a more precise lower bound on the absolute indicator by using the fact that all autocorrelation coefficients are divisible by~$8$.
We even have a stronger property for functions having a lower algebraic degree as shown in the following proposition.
\begin{proposition}[Divisibility]\label{prop:act:divisible-by-eight}
    Let $n > 2$ and $F : \F_2^n \to \F_2^n$ be a permutation with algebraic degree at most~$d$.
    Then, for any $\alpha, \beta \in \F_2^n$,
    $\AC{F}(\alpha, \beta)$ is divisible by $2^{\ceil{\frac{n}{d-1}}+1}$.

    Most notably, the autocorrelation coefficients of a permutation are divisible by~$8$.
\end{proposition}
\begin{proof}
    From the definition of the autocorrelation, we know that
    \begin{equation*}
        \AC{F}(\alpha, \beta) = \FC{\derive{\alpha}{F_\beta}}(0).
    \end{equation*}
    For the sake of readability, we define $h_{\alpha, \beta} = \derive{\alpha}{F_\beta}$.
    We can derive two properties of this Boolean function $h_{\alpha, \beta}$.
    First, as $F$ has degree at most~$d$, $\deg(h_{\alpha, \beta}) \leqslant d-1$.
    Second, $h_{\alpha, \beta}(x) = h_{\alpha, \beta}(x+\alpha)$.

    We now focus on the divisibility of $\FC{h_{\alpha, \beta}}(0)$.
    First, assume for simplicity that $\alpha = e_n$, we discuss the general case afterwards.
    Then we can write $h_{e_n, \beta}$ as $h_{e_n, \beta}(x) = g(x_1, \ldots, x_{n-1})$ with $g : \F_2^{n-1} \to \F_2$, because $h_{e_n, \beta}(x+e_n) = h_{e_n, \beta}(x)$.
    The Walsh coefficient of $h_{e_n, \beta}$ at point~$0$ can then be computed as
    \begin{equation*}
        %\label{act:eq:f-fc}
        \FC{h_{e_n, \beta}}(0)
        = \sum_{x \in \F_2^{n-1}, x_n \in \F_2} \parens{-1}^{f(x, x_n)}
        = 2 \cdot \sum_{x \in \F_2^{n-1}} \parens{-1}^{g(x)}
        = 2 \cdot \FC{g}(0)
    \end{equation*}
    Now $\deg(g) \leqslant d-1$.
    It is well-known that the Walsh coefficients of a Boolean function~$f$ are divisible by $2^{\ceil{\frac{n}{\deg(f)}}}$ (see~\cite{DM:McE72} or \cite[Section~3.1]{BMM:Carlet10a}).
    We then deduce that $\FC{g}(0)$ is divisible by $2^{\ceil{\frac{n-1}{d-1}}}$, implying that $\FC{h_{e_n, \beta}}(0)$ is divisible by $2^{\ceil{\frac{n-1}{d-1}}+1}$.
    Most notably, if $F$ is bijective, $d \leqslant n-1$.
    We then have that
    \begin{equation*}
        \ceil{\frac{n-1}{d-1}} + 1 \geqslant 3,
    \end{equation*}
    implying that $\FC{h_{e_n, \beta}}(0)$ is divisible by~$8$.

    In the case that $\alpha \neq e_n$, we can find a linear transformation $L$, such that $L(e_n) = \alpha$, with which we have the affine equivalent function $G = F \circ L \aequiv F$.
    Now for $G$ the same argument as above holds and thus $\AC{G}(\alpha, \beta)$ is divisible by $2^{\ceil{\frac{n}{d-1}}+1}$.
    Due to the affine invariance of $G$'s and $F$'s \ACT/ spectra the same holds for $\AC{F}(\alpha, \beta)$ in this case.
\end{proof}
The absolute indicator is known for very few permutations only, except in the case of permutations of degree less than or equal to~$2$, where the result is trivial.
To our best knowledge, one of the only functions whose absolute indicator is known is the inverse mapping $F(x)=x^{2^n-2}$ over $\F_{2^n}$~\cite{FF:ChaHelZin07}: $\absoluteindicator{F} = 2^{\frac{n}{2}+1}$ when $n$ is even.
When $n$ is odd, $\absoluteindicator{F}=\mathcal{L}(F)$ when $\mathcal{L}(F) \equiv 0 \bmod{8}$, and $\absoluteindicator{F}=\mathcal{L}(F) \pm 4$ otherwise (see \cref{sec:APN functions} for an alternative proof).

We now study the absolute indicator of some types of vectorial Boolean functions.

\subsection{Case of power permutations}
For power permutations on $\F_{2^n}$ we can show that the autocorrelation spectrum is invariant for all component functions, analogously to the same well known fact for the Walsh spectrum.
In this case, the trace function is used as the inner product, \ie/ $\iprod{\alpha}{\beta} = \tr_n(\alpha \beta)$ with
\begin{align*}
    \tr_n : \F_{2^n} &\to \F_2 \\
    \tr_n(x) &\coloneqq \sum_{i=0}^{n-1} x^{2^i}
\end{align*}
where we leave out the subscript, if it is clear from the context.
Thus we have the following corollary.

\begin{corollary}
    Let $F$ be a power permutation on $\F_{2^n}$, \ie/ $F = x^k$ where for $k$ it holds that $\gcd(k, 2^n-1)=1$.
    Then, for all non-zero $\alpha$ and $\beta$ in $\F_{2^n}$,
    \begin{equation*}
        \AC{F}(\alpha, \beta) = \AC{F}(1, \alpha^k \beta) = \AC{F}(\alpha \beta^{\frac{1}{k}}, 1)\;.
    \end{equation*}
    Most notably, all non-zero component functions $F_\beta$ of $F$ have the same (Boolean) absolute indicator: $\absoluteindicator{F} = \absoluteindicator{F_\beta}$ for all $\beta \in \F_{2^n} \setminus \set{0}$ and $\absoluteindicator{F_\beta} > 2^{n/2}$.
\end{corollary}
\begin{proof}
    The fact that $\AC{F}(\alpha, \beta) = \AC{F}(1, \alpha^k \beta)$ has been proved for instance in~\cite[Proposition~4]{TIT:BCCL06}.
    The second equality comes from the fact that
    \begin{equation*}
        \alpha \cdot \parens{x^k + \parens{x+\alpha}^k} = 1 \cdot \parens{\parens{\beta^{\frac{1}{k}}x}^k + \parens{\beta^{\frac{1}{k}}x + \alpha \beta^{\frac{1}{k}}}^d},
    \end{equation*}
    where $\beta^{\frac{1}{k}}$ only exists if $\gcd(k, 2^n-1) = 1$.
    It follows that it is enough to compute only one column of the \ACT/, as the remaining ones are just permutations of each other.
    In other words, all Boolean functions $F_\beta$, $\beta \neq 0$, have the same absolute indicator.
\end{proof}

\subsection{Case of APN functions}\label{sec:APN functions}
In the specific case of \APN/ functions, we can also exhibit a stronger condition than \cref{prop:lowerbound} on the lowest possible absolute indicator.
\begin{proposition}[Lowest possible indicator for APN functions]\label{prop:APN}
    Let $n$ be a positive integer.
    If there exists an \APN/ function from $\F_2^n$ to $\F_2^n$ with absolute indicator~$M$, then there exists a balanced Boolean function of $n$~variables with linearity~$M$.
\end{proposition}
\begin{proof}
    If $F$ is \APN/, then $\DC{F}(\alpha, \beta) \in \set{0,2}$ for all $\alpha, \beta$, $\alpha \neq 0$.
    It follows that, for each nonzero~$\alpha$, we can define a Boolean function $g_\alpha$ of $n$~variables such that
    \begin{equation*}
        g_\alpha(\beta) = \begin{cases*}
            0 & if $\DC{F}(\alpha, \beta)=0$ \\
            1 & if $\DC{F}(\alpha, \beta)=2$\;.
        \end{cases*}
    \end{equation*}
    Equivalently,
    \begin{equation*}
        \DC{F}(\alpha, \beta) = 1 - (-1)^{g_\alpha(\beta)}\;.
    \end{equation*}
    Obviously, all $g_\alpha$ are balanced.
    Moreover, we deduce from \cref{eq2} that, for all nonzero $\alpha, \beta$,
    \begin{align*}
        \AC{F}(\alpha, \beta)
        &= \sum_{\theta \in \F_2^n} (-1)^{\iprod{\beta}{\theta}} \DC{F}(\alpha,\theta) \\
        &= \sum_{\theta \in \F_2^n} (-1)^{\iprod{\beta}{\theta}} \parens{1 - (-1)^{g_\alpha(\theta)}}\\
        &= -\sum_{\theta \in \F_2^n} (-1)^{\iprod{\beta}{\theta} + g_\alpha(\theta)} = -\FC{g_\alpha}(\beta)
    \end{align*}
    where $\FC{g_\alpha}(\beta)$ denotes the value of the Walsh transform of $g_\alpha$ at point~$\beta$.
    This implies that
    \begin{equation*}
        \max_{\beta \neq 0} \abs{\AC{F}(\alpha, \beta)} = \linearity{g_\alpha}\;.
    \end{equation*}
    The result then directly follows.
\end{proof}
To our best knowledge, the smallest known linearity for a balanced Boolean function is obtained by Dobbertin's recursive construction~\cite{FSE:Dobbertin94}.
For instance, for $n=9$, the smallest possible linearity for a balanced Boolean function is known to belong to the set $\set{24, 28, 32}$, which implies that exhibiting an \APN/ function over $\F_2^9$ with absolute indicator~$24$ would determine the smallest linearity for such a function.

It is worth noticing that the proof of the previous proposition shows that the knowledge of $g$ directly determines the \ACT/.
This explains why the absolute indicator of the inverse mapping over $\F_{2^n}$, $n$ odd, is derived from its linearity as proved in~\cite[Theorem~1]{FF:ChaHelZin07} and detailed in the following example.
\begin{example}[ACT of the inverse mapping, $n$ odd]
    For any $\alpha \in \F_{2^n} \setminus \set{0}$, the Boolean function $g_\alpha$ which characterises the support of Row~$\alpha$ in the \DDT/ of the inverse mapping $F:x \mapsto x^{-1}$ coincides with $(1+F_{\alpha^{-1}})$ except on two points:
    \begin{equation*}
        g_\alpha(\beta) = \begin{cases*}
            1+\tr(\alpha^{-1}\beta^{-1}) & if $\beta \not \in \set{0, \alpha^{-1}}$\\
            0                            & if $\beta = 0$\\
            1                            & if $\beta = \alpha^{-1}$
        \end{cases*}\;.
    \end{equation*}
    This comes from the fact that the equation
    \begin{equation*}
        (x+\alpha)^{-1} + x^{-1} = \beta
    \end{equation*}
    for $\beta \neq \alpha^{-1}$ can be rewritten as
    \begin{equation*}
        x + (x+\alpha) = \beta (x+\alpha)x
    \end{equation*}
    or equivalently when $\beta \neq 0$, by setting $y=\alpha^{-1}x$,
    \begin{equation*}
        y^2+y = \alpha^{-1}\beta^{-1}\;.
    \end{equation*}
    It follows that this equation has two solutions if and only if $\tr(\alpha^{-1}\beta^{-1}) = 0$.
    From the proof of the previous proposition, we deduce
    \begin{align*}
        \AC{F}(\alpha, \beta)
        &= -\FC{g_\alpha}(\beta)  \\
        &= \FC{F_{\alpha^{-1}}}(\beta) + 2\parens{1 - (-1)^{\tr(\alpha^{-1}\beta)}}\;,
    \end{align*}
    where the additional term corresponds to the value of the sum defining the Walsh transform $\FC{F_{\alpha^{-1}}}(\beta)$ at points~$0$ and $\alpha^{-1}$.
\end{example}

It can be observed that \cref{prop:boundAPN,prop:APN} are actually more general and apply as soon as the \DDT/ of $F$ contains one row composed of $0$s and $2$s only.
\begin{proposition}
    Let $F$ be a function from $\F_2^n$ into $\F_2^n$.
    Then, for any fixed $\alpha \in \F_2^n \setminus \set{0}$, the following properties are equivalent:
    \begin{enumerate}
        \item[(i)] $\displaystyle \sum_{\beta \in \F_2^n \setminus \set{0}} \AC{F}^2(\alpha, \beta) = 2^{2n}$
        \item[(ii)] For all $\beta \in \F_2^n$, $\DC{F}(\alpha, \beta) \in \set{0,2}$.
    \end{enumerate}
    Moreover, if these properties hold, then there exists a balanced Boolean function $g: \F_2^n \rightarrow \F_2$ such that
    \begin{equation*}
        \linearity{g} = \max_{\beta \in \F_2^n \setminus \set{0}} \abs{\AC{F}(\alpha, \beta)}\;.
    \end{equation*}
\end{proposition}

Let us now take a closer look at the absolute indicator of some specific \APN/ permutations.

\subsection{Case of APN permutations}\label{sec:APN}

As previously observed, the \ACT/ and the absolute indicator are not invariant under inversion.
Then, while the absolute indicator of a quadratic permutation is trivially equal to $2^{n}$, computing the absolute indicator of the inverse of a quadratic permutation is not obvious at all.
Indeed, the absolute indicator depends on the considered function, as we will see next.

\newthoughtpar{Inverses of quadratic \APN/ permutations, $n$ odd}
For instance, for $n=9$, the inverses of the two \APN/ Gold permutations $x^3$ and $x^5$, namely $x^{341}$ and $x^{409}$, do not have the same absolute indicator: the absolute indicator of $x^{341}$ is~$56$ while the absolute indicator of $x^{409}$ is~$72$.

Nevertheless, the specificity of quadratic \APN/ permutations for $n$ odd is that they are \emph{crooked}~\cite{EJC:BenFon98}, which means that the image sets of their derivatives $\derive{\alpha}{F}$, $\alpha \neq 0$, is the complement of a hyperplane $\Span{\pi(\alpha)}^\perp$.
Moreover, it is known (see \eg/~\cite[Proof of Lemma~5]{TIT:CanCha03}) that all these hyperplanes are distinct, which implies that $\pi$ is a permutation of $\F_2^n$ when we add to the definition that $\pi(0)=0$.
Then, the following proposition shows that, for any quadratic \APN/ permutation~$F$, the \ACT/ of $F^{-1}$ corresponds to the Walsh transform of $\pi$.

\begin{proposition}\label{prop:inverse quadratic}
    Let $n$ be an odd integer and $F$ be a quadratic \APN/ permutation over $\F_2^n$.
    Let further $\pi$ be the permutation of $\F_2^n$ defined by
    \begin{equation*}
        \Im{\derive{\alpha}{F}} = \F_2^n \setminus \Span{\pi(\alpha)}^\perp, \text{when } \alpha \neq 0\;,
    \end{equation*}
    and $\pi(0)=0$.
    Then, for any nonzero $\alpha$ and $\beta$ in $\F_2^n$, we have
    \begin{equation*}
        \AC{F^{-1}}(\alpha, \beta) = - \FC{\pi}(\beta,\alpha)\;.
    \end{equation*}
    It follows that
    \begin{equation*}
        \absoluteindicator{F^{-1}} \geqslant 2^{\frac{n+1}{2}}
    \end{equation*}
    with equality if and only if $\pi$ is an \AB/ permutation.
\end{proposition}
\begin{proof}
    Let $\alpha$ and $\beta$ be two nonzero elements in $\F_2^n$.
    Then, from Relation~\eqref{eq2}, we deduce
    \begin{align*}
        \AC{F^{-1}}(\alpha, \beta)
        &= \sum_{\gamma \in \F_2^n} (-1)^{\iprod{\beta}{\gamma}} \DC{F^{-1}}(\alpha,\gamma)\\
        &= \sum_{\gamma \in \F_2^n} (-1)^{\iprod{\beta}{\gamma}} \DC{F}(\gamma,\alpha)\;.
    \end{align*}
    By definition of $\pi$, we have that, for any nonzero $\gamma$,
    \begin{equation*}
        \DC{F}(\gamma,\theta) = \begin{cases*}
            2 & if $\iprod{\theta}{\pi(\gamma)} = 1$\\
            0 & if $\iprod{\theta}{\pi(\gamma)} = 0$
        \end{cases*}\;.
    \end{equation*}
    It then follows that
    \begin{equation*}
        \DC{F}(\gamma,\theta) = 1 - (-1)^{\iprod{\pi(\gamma)}{\theta}}
    \end{equation*}
    where this equality holds for all $(\gamma, \theta) \neq (0,0)$ by using that $\pi(0)=0$.
    Therefore, we have, for any nonzero $\alpha$ and $\beta$,
    \begin{equation*}
        \AC{F^{-1}}(\alpha, \beta) = \sum_{\gamma \in \F_2^n} (-1)^{\iprod{\beta}{\gamma}} \parens{1 - (-1)^{\iprod{\pi(\gamma)}{\alpha}}} = -\FC{\pi}(\beta, \alpha)\;.
    \end{equation*}
    As a consequence, $\absoluteindicator{F^{-1}}$ is equal to the linearity of $\pi$, which is at least $2^{\frac{n+1}{2}}$ with equality for \AB/ functions.
\end{proof}

It is worth noticing that the previous proposition is valid, not only for quadratic \APN/ permutations, but for all \emph{crooked} permutations, which are a particular case of \AB/ functions.
However, the existence of crooked permutations of degree strictly higher than~$2$ is an open question.

As a corollary of the previous proposition, we get some more precise information on the autocorrelation spectrum of the quadratic power permutations corresponding to Gold exponents, \ie/ $F(x)=x^{2^i+1}$.
Recall that $x^{2^i+1}$ and $x^{2^{n-i}+1}$ are affine equivalent since the two exponents belong to the same cyclotomic coset modulo~$(2^n-1)$.
This implies that their inverses share the same autocorrelation spectrum.

\begin{corollary}
    Let $n > 5$ be an odd integer and $0 < i < n$ with $\gcd(i,n)=1$.
    Let $F$ be  the \APN/ power permutation over $\F_{2^n}$ defined by $F(x)=x^{2^i+1}$.
    Then, for any nonzero $\alpha$ and $\beta$ in $\F_2^n$, we have
    \begin{equation*}
        \AC{F^{-1}}(\alpha, \beta) = -\FC{\pi}(\beta, \alpha) \quad \text{where} \quad \pi(x) = x^{2^n-2^i-2}\;.
    \end{equation*}
    Most notably, the absolute indicator of $F^{-1}$ is strictly higher than $2^{\frac{n+1}{2}}$.
\end{corollary}
\begin{proof}
    The result comes from the form of the function $\pi$ which defines the \DDT/ of $x^{2^i+1}$.
    Indeed, for any nonzero $\alpha \in \F_{2^n}$, the number $\DC{F}(\alpha, \beta)$ of solutions of
    \begin{equation*}
        (x + \alpha)^{2^i+1} + x^{2^i+1} = \beta
    \end{equation*}
    is equal to the number of solutions of
    \begin{equation*}
        x^{2^i} + x = 1 + \beta \alpha^{-(2^i+1)}
    \end{equation*}
    which is nonzero if and only if $\tr(\beta\alpha^{-(2^i+1)})=1$.
    It follows that
    \begin{equation*}
        \pi(x) = x^{2^n-2^i-2}\;.
    \end{equation*}
    The autocorrelation spectrum of $F^{-1}$ then follows from \cref{prop:inverse quadratic}.
    Moreover, this function $\pi$ cannot be \AB/ since \AB/ functions have algebraic degree at most~$\frac{n+1}{2}$~\cite[Theorem~1]{DCC:CarChaZin98}, while $\pi$ has degree~$(n-2)$.
    It follows that $\pi$ cannot be \AB/ when $n > 5$.
    Therefore, the absolute indicator of the inverse of $F^{-1}$, \ie/ the linearity of $\pi$, is strictly higher than $2^{\frac{n+1}{2}}$.
\end{proof}
In the specific case $n=5$, it can easily be checked that the inverses of all Gold APN permutations $F(x)=x^{2^i+1}$ have absolute indicator~$8$.

\newthoughtpar{Cubic APN permutations}
In the case of APN permutations of degree~$3$, we have a more precise result.

\begin{proposition}[Cubic APN permutations] \label{prop:cubic}
Let $F: \F_2^n \rightarrow \F_2^n$ be \APN/ with degree~$3$.
Then we have that for non-zero $\alpha$ and $\lambda$
\begin{equation*}
    \abs{\AC{F}(\alpha,\lambda)} \in \set{0, 2^{\frac{n+d(\alpha,\lambda)}{2}}}\;,
\end{equation*}
where $d(\alpha,\lambda) = \dim \LS(\derive{\alpha}{F_\lambda}) = \dim \set{\beta \given \derive{\alpha,\beta}{F_\lambda} = c}$ and $c \in \F_2$ is constant.
Moreover, $\AC{F}(\alpha,\lambda) = 0$ if and only if $\derive{\alpha}{F_\lambda}$ is balanced, which equivalently means that it has an all-one derivative.
\end{proposition}

From this proposition, if $n$ is odd, we obviously have that $\absoluteindicator{F} \geqslant 2^{\frac{n+1}{2}}$ with equality if and only if, for any nonzero $\alpha,\lambda \in \F_2^n$, either $\LS(\derive{\alpha}{F_\lambda})=\set{0,\alpha}$ or there exists $\beta$ such that $\derive{\beta}{\derive{\alpha}{F_\lambda}} = 1$.
Moreover, if $F$ is \APN/ and $\absoluteindicator{F} = 2^{\frac{n+1}{2}}$, it follows from \cref{prop:boundAPN} that the number of nonzero $\lambda$ such that $\LS(\derive{\alpha}{F_\lambda}) = \set{0,\alpha}$ is exactly $2^{n-1}$.

Additionally, an upper bound on the absolute indicator can be established for two cubic APN permutations, namely the first Kasami power function and the Welch function.
We denote the Kasami power functions $K_i$ and the Welch power function $W$ by
\begin{equation*}
    \begin{aligned}
        K_i : \F_{2^n} &\to \F_{2^n} \\
        K_i : x &\mapsto x^{(2^{3i}+1)/(2^i+1)} \\ &= x^{4^i-2^i+1}
    \end{aligned}
    \qquad \text{and} \qquad
    \begin{aligned}
        W : \F_{2^n} &\to \F_{2^n} \\
        W : x &\mapsto x^{2^{(n-1)/2}+3}\;.
    \end{aligned}
\end{equation*}

\begin{proposition}[\citeonly{TIT:Carlet08}, Lemma 1]
    The absolute indicator for $W$ on $\F_{2^n}$ is bounded from above by
    \begin{equation*}
        \absoluteindicator{W} \leqslant 2^{\frac{n+5}{2}}
    \end{equation*}
\end{proposition}

As long as the (regular) degree of the derivatives is small compared to the field size, the Weil bound gives a nontrivial upper bound for the absolute indicator of a vectorial Boolean function.
This is particularly interesting for the Kasami functions as the Kasami exponents do not depend on the field size (contrary to for example the Welch exponent).
\begin{proposition}
    The absolute indicator of $K_i$ on $\F_{2^n}$ is bounded from above by
    \begin{equation*}
        \absoluteindicator{K_i} \leqslant (4^i-2^{i+1}) \cdot 2^{\frac{n}{2}}.
    \end{equation*}
    In particular,
    \begin{equation*}
        \absoluteindicator{K_2} \leqslant 2^{\frac{n+5}{2}}.
    \end{equation*}
\end{proposition}
\begin{proof}
    Note that the two exponents with the highest degree of any derivative of $K_i$ are $4^i-2^i$ and $4^i-2^{i+1}+1$.
    The first exponent is even, so it can be reduced using the relation $\tr(y^2)=\tr(y)$.
    The result then follows from the Weil bound.
    Combining the bound with \cref{prop:cubic} yields the bound on $K_2$.
\end{proof}
In the two cases of $W$ and $K_2$, we deduce that the absolute indicator belongs to $\set{2^{\frac{n+1}{2}}, 2^{\frac{n+3}{2}}, 2^{\frac{n+5}{2}}}$.
We actually conjecture the following.
\begin{conjecture}\label{con:kasami_welch}
    Let $n \geqslant 9$ be odd.
    Then $\displaystyle\absoluteindicator{K_2} \geqslant 2^{\frac{n+3}{2}}$ and $\displaystyle\absoluteindicator{W} \geqslant 2^{\frac{n+3}{2}}$.
\end{conjecture}

\begin{conjecture}\label{con:kasami}
    If $n$ odd and $n \not \equiv 0 \bmod{3}$, then $\absoluteindicator{K_i} = 2^{\frac{n+1}{2}}$.
\end{conjecture}

Some other results on the autocorrelations of the Boolean functions $\tr(x^k)$ are known in the literature, which can be trivially extended to the vectorial power functions $x^k$ if $\gcd(k,n)=1$, see~\cite[Theorem~5]{SAC:GonKho03}, \cite{TIT:Carlet08} and \cite[Lemmata 2 and 3]{IS:SunWu09}.
In the case $n = 6r$ and $k=2^{2r}+2^r+1$, the power monomial $x^k$ is not a permutation, but results for all component functions of $x^k$ were derived by \textcite{FFA:CanChaKyu08}.
We summarise the results in the following proposition.

\begin{proposition}
    Let $F(x)=x^k$ be a function on $\F_{2^n}$.
    \begin{enumerate}
        \item If $n$ is odd and $k=2^r+3$ with $r=\frac{n+1}{2}$ then $\absoluteindicator{F}\in \{2^\frac{n+1}{2},2^\frac{n+3}{2}\}$.
        \item If $n$ is odd and $k$ is the $i$-th Kasami exponent, where $3i \equiv \pm 1 \pmod n$, then $\absoluteindicator{F}=2^\frac{n+1}{2}$.
        \item If $n=2m$ and $k=2^{m+1}+3$ then $\absoluteindicator{F}\leqslant 2^{\frac{3m}{2}+1}$.
        \item If $n=2m$, $m$ odd and $k=2^{m}+2^{\frac{m+1}{2}}+1$ then $\absoluteindicator{F}\leqslant 2^{\frac{3m}{2}+1}$.
        %\item If $n=4r$ and $k=2^{2r}+2^{r}+1$ then $\absoluteindicator{F}\leqslant 2^{3r+1}$. % correct, but needs explanation
        \item If $n=6r$ and $k=2^{2r}+2^r+1$ then $\absoluteindicator{F}=2^{4r}$.
    \end{enumerate}
\end{proposition}

We now provide a different proof of the second case in the previous proposition that additionally relates the autocorrelation table of $K_i$ with the Walsh spectrum of a Gold function.

\begin{proposition}[\citeonly{DCC:Dillon99}]\label{thm:support}
    Let $n$ be odd, not divisible by $3$ and $3i \equiv \pm 1 \pmod n$.
    Set $f=\tr(x^k)$ where $k=4^i-2^i+1$ is the $i$-th Kasami exponent.
    Then
    \begin{equation*}
        \supp{\FC{f}} = \set{\alpha \given \tr(\alpha^{2^k+1})=1}\;.
    \end{equation*}
\end{proposition}

\begin{proposition}\label{prop:kasami}
    Let $n$ be odd, not divisible by $3$ and $3i \equiv \pm 1 \pmod n$.
    Then
    \begin{equation*}
        \AC{K_i}(\alpha, \beta) = -\sum_{\gamma \in \F_{2^n}} (-1)^{\tr(\alpha \beta^{1/k} \gamma + \gamma^{2^k+1})},
    \end{equation*}
    where $k = 4^i-2^i+1$ is the $i$-th Kasami exponent and $1/k$ denotes the inverse of $k$ in $\Z_{2^n-1}$.
    In particular, $\absoluteindicator{K_i}=2^\frac{n+1}{2}$.
\end{proposition}
\begin{proof}
    It is well-known that, if $F$ is a power permutation over a finite field, its Walsh spectrum is uniquely defined by the entries $\FC{F}(1,\beta)$.
    Indeed, for $\beta \neq 0$,
    \begin{align*}
        \FC{K_i}(\gamma,\beta)&=\sum_{x \in \F_{2^n}} (-1)^{\tr (\beta x^k + \gamma x)} \\
        &= \sum_{x \in \F_{2^n}} (-1)^{\tr (x^k + \gamma \beta^{-1/k} x)} = \FC{K_i}(1,\gamma \beta^{-1/k}) \in \set{0, \pm 2^\frac{n+1}{2}}\;,
    \end{align*}
    where the last fact follows because the Kasami function is \AB/.
    Then, by  \cref{eq1} and \cref{thm:support}, for any nonzero $\alpha$ and $\beta$,
    \begin{align}
           \AC{K_i}(\alpha, \beta)
        &= 2^{-n} \sum_{\gamma \in \F_{2^n}} (-1)^{\tr(\alpha \gamma)}\FC{K_i}^2(\gamma, \beta)
         = 2^{-n} \sum_{\gamma \in \F_{2^n}} (-1)^{\tr(\alpha \gamma)}{\FC{K_i}(1, \gamma \beta^{-1/k})}^2 \nonumber \\
        &= 2^{-n} \sum_{\gamma \beta^{-1/k} \in \supp{\FC{\tr(x^k)}}} (-1)^{\tr(\alpha \gamma)} 2^{n+1}
         = 2 \sum_{\gamma \in B} (-1)^{\tr(\alpha \gamma)}\;,\label{eq:kasami}
        \intertext{where $B = \set{\gamma \in \F_{2^n} \given \tr((\gamma \beta^{-1/k})^{2^k+1}) = 1}$.
                   We have
                   \begin{align*}
                       0 &= \sum_{\gamma \notin B} (-1)^{\tr(\alpha \gamma)} + \sum_{\gamma \in B} (-1)^{\tr(\alpha \gamma)} \\
                         &= \sum_{\gamma \notin B} (-1)^{\tr(\alpha \gamma + (\gamma \beta^{-1/k})^{2^k+1})} - \sum_{\gamma \in B} (-1)^{\tr(\alpha \gamma + (\gamma \beta^{-1/k})^{2^k+1})}\;,
                   \end{align*}
                   so
                   \begin{align*}
                          \sum_{\gamma \in \F_{2^n}} (-1)^{\tr(\alpha \gamma+(\gamma \beta^{-1/k})^{2^k+1})}
                       &= 2\sum_{\gamma \in B} (-1)^{\tr(\alpha \gamma+(\gamma \beta^{-1/k})^{2^k+1})}
                        = -2\sum_{\gamma \in B} (-1)^{\tr(\alpha \gamma)}\;.
                   \end{align*}
                   Plugging this into \cref{eq:kasami}, we obtain
                  }
           \AC{K_i}(\alpha, \beta)
        &= -\sum_{\gamma \in \F_{2^n}} (-1)^{\tr(\alpha \gamma + (\gamma \beta^{-1/k})^{2^k+1})}
         = -\sum_{\gamma \in \F_{2^n}} (-1)^{\tr(\alpha \beta^{1/k} \gamma + \gamma^{2^k+1})}\;. \nonumber
    \end{align}
    It can be easily checked that the equation also holds for $\beta=0$.
    Observe that $\gcd(k,n)=1$, so the Gold function is \AB/ and
    \begin{equation*}
        \absoluteindicator{K_i} = 2^{\frac{n+1}{2}}\;.
    \end{equation*}
\end{proof}
Note that the cases $3i \equiv 1 \pmod n$ and $3i \equiv -1 \pmod n$ are essentially only one case because the $i$-th and $(n-i)$-th Kasami exponents belong to the same cyclotomic coset.
Indeed, $(4^{(n-i)}-2^{n-i}+1)2^{2i} \equiv 4^{i}-2^i+1 \pmod {2^n-1}.$

\newthought{Note}\hspace{1.5em}
\textcite{ARXIV:LLLQ19} independently observed similar results.
In \citeonly{CKLLLQW19} we merged both drafts; the paper is currently under submission.
